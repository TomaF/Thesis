%% This is an example first chapter.  You should put chapter/appendix that you
%% write into a separate file, and add a line \include{yourfilename} to
%% main.tex, where `yourfilename.tex' is the name of the chapter/appendix file.
%% You can process specific files by typing their names in at the 
%% \files=
%% prompt when you run the file main.tex through LaTeX.
\chapter{Introduction}

Electronic voting (e-voting) is a term used to describe the act of voting using electronic systems. Generally, the term e-voting is applicable for two different types of electronic voting scenarios:  1) 'Remote e-voting' voting over the Internet via personal gadgets (laptops, smartphones, etc.) at any place outside the polling station and  2) 'Polling place e-voting' -- voting inside a polling station or similar premises controlled by electoral staff. In this work, both meanings are used.\\

Electronic voting has been in the centre of researchers attention for over the last twenty years. Up to now, many e-voting schemes with quite strong security guarantee have been proposed. Perhaps the most well known and studied system is Helios designed by Ben Adida \cite{Adida2008}.  Helios is an open-source purely cryptographic voting protocol which does not rely on paper. Other examples of purely electronic systems are Demos-2 \cite{Kiayias2015} and Civitas \cite{Clarkson2008}.  Another subclass of e-voting systems is so-called hybrid systems where paper ballots are used for computing tally or ensuring the integrity of an election. Demos \cite{Kiayias2015a}, ThreeBallot \cite{Rivest2006}, Pr\^{e}t-\`a-Voter \cite{Ryan2006} and Scantegrity \cite{Chaum2009} are examples of hybrid systems.  \\

Nowadays, some countries allow their citizens who are living or staying abroad to vote remotely. However, only a few countries allow external voters to cast their votes electronically. For the last sixty years e-voting (including remote e-voting) has been conducted at least once only in the following countries Australia, Belgium, Brazil, Canada, Estonia, France, Germany, India, Italy, the Netherlands (Rijnland Internet Election System), Norway, Peru, Romania, Switzerland, the UK, Venezuela, and the Philippines.\\

Benefits of using electronic voting  are significant:
\begin{enumerate}
\item Voting is easier and more convenient;
\item E-voting system can be completely auditable. Therefore elections are completely transparent;
\item Results are announced faster;
\item Increased engagement and turnout; 
\item Increased accessibility;
\item Voting is provably secure.
 \end{enumerate}

However, all of the potential benefits are moot if we can not trust the election results. Some argue that electronic voting is not secure because of issues with the technology, vast possibilities of fraud, and protection of voters privacy.  To eliminate the security risks a number of security requirements for e-voting systems were developed: ensuring one vote per voter, voters eligibility, maintaining voter anonymity, the accuracy of tallying and prevention of fraud.\\
 
Among the security properties that have been identified for e-voting, there are two highly desirable properties that are considered to be crucial and has been investigated extensively: E2E Verifiability and Voter Privacy.\\

 E2E Verifiability means that it is possible to verify the correctness of the election tally based on feedback from the participants and by examining the public election transcript. Informally the E2E verifiable property means that any voter can detect that the election outcome has been manipulated.  Formally, E2E verifiability is defined as the ability of a voter  to verify that: 1) his vote was properly cast, 2) recorded as cast and 3) tallied into the election result as recorded. \\
 
 Voter Privacy can be described as follows: e-voting system should not reveal how a particular voter voted. Voter privacy can be divided into tree levels of different strength (in increasing order):
\begin{enumerate}
\item Ballot privacy: A set of ballots would not reveal  voters' choices to anyone \cite{Bernhard2015}. 
\item Receipt-freeness: An honest voter cannot prove to an adversary that he voted in a certain way \cite{Kremer}.
\item Coercion resistance: A corrupted voter cannot prove to an adversary that he voted in a certain way \cite{Delaune2006}.
\end{enumerate}

It has been observed that voter privacy and E2E verifiability requirements inherently contradict each other at some point. In this work, we perform a thorough and formal study on privacy in E2E verifiable e-voting systems to analyse those restrictions and define the privacy limits. We suggest a stronger privacy definition, that does not impose restrictions on an adversarial behaviour. Also, we define a notion of strict privacy and prove in simulation-based settings that it contradicts E2E Verifiability. At the final chapter of this work, we present a blind signature scheme, that captures the idea of voting anonymously, and prove that the whole class of such systems is not E2E Verifiable.  

\section{Preliminaries}
Through this paper, security parameter is denoted as $\lambda$. The notion $negl(\lambda)$ is used to denote a negligible function in $\lambda$, i.e., it always holds that $negl(\lambda) < \frac{1}{\lambda^c}$ for any $0<c \in \mathbb{Z}$ for sufficient large $\lambda$. \\

Let $\Pi$ be an e-voting system and $\mathcal{P} = \{P_1,\dots,P_m\}$ be a set of $m$ candidates. Voters $\mathcal{V} = \{V_1,\dots,V_n\}$ use $\Pi$ to vote for some allowed subset of candidates selections from the collection of allowed selections $\mathcal{U}$ (which may include a 'blank' option too). \\
 
 In this modelling, the election system involves five types of entities, the voters $V_1, \dots , V_n$, possibly equipped with the voting supporting device ($\vsd$) and the auditing supporting device ($\asd$), the election authority ($\ea$), the vote collector ($\voc$), the trustee ($T$), and the bulletin board ($\bb$) whose only role is to provide storage for the election transcript for the purpose of verification. 
\begin{enumerate}
\item $\bb$ is completely passive and is only writeable by the $\voc$, $T$ and $\ea$ and readable by anyone. 
\item Voters submit their votes by engaging in the ballot casting protocol to the $\voc$ and they are not allowed to interact with each other. 
\item $\voc$ only role is to collect votes and write them to $\bb$
\item $T$ is responsible for computing the tally and announcing the election result.
\item $\ea$ - prepares all the election setup information and distributes the voters' ballots.
\end{enumerate}

In many election systems, the $\ea$ and $T$ are implemented by more than a single authority.  However, since  the system is considered to be malicious as a whole, this is completely immaterial. Hence, for simplicity in the syntax, we assume that both $\ea$ and $T$ are single entities. 

\subsection{Syntax and Correctness} 
\begin{figure}
\begin{tikzpicture}[scale=0.8]
  %\draw[step=10pt,gray,very thin] (0,0) grid (30,30);
  
  %\draw[step=1pt,thick] (25,15) rectangle (28,16.5) ;
  \node[draw, align = center, text width = 60pt, text height=40pt,rounded corners=2.5pt,line width=2pt,fill = white, drop shadow={shadow xshift=3.5pt, shadow yshift=-2pt, fill=mygray}]  (EA) at (7,24) {};
  
    \node[draw, align = center, text width = 60pt, text height=40pt,rounded corners=2.5pt,line width=2pt,fill = white, drop shadow={shadow xshift=3.5pt, shadow yshift=-2pt, fill=mygray}]  (VC) at (10,20) {};
    
      \node[draw, align = center, text width = 60pt, text height=40pt,rounded corners=2.5pt,line width=2pt,fill = white, drop shadow={shadow xshift=3.5pt, shadow yshift=-2pt, fill=mygray}]  (T) at (5,20) {};
    
  \node at (EA) {\Large{\textbf{EA}}};
 \node at (VC) {\Large{\textbf{VC}}};
  \node at (T) {\Large{\textbf{T}}};
%



 %
% \node (CD) at (11,23.5) {};

 
\node[draw, fill =white, align = center, text width = 85pt,text height=50pt,rounded corners=3pt,line width=2.2pt, drop shadow={shadow xshift=3.5pt, shadow yshift=-3pt, fill=mygray}] (BB) at (10,16) {}; 

 \node[below right, inner sep=6pt] at (BB.north west) {\Large{\textbf{BB}}};
 
  \draw[->,>=stealth,line width=2pt] (VC)--(BB) ;
    \draw[->,>=stealth,line width=2pt] (EA.south) |- (BB.west);
    \draw[->,>=stealth,line width=2pt] (T.south) |- (BB.west);

 
 %
  \node[SD](VSD1) at (13,22.75) {};
        \node[SD](VSD2) at (13,20.75) {};
     \node[SD](VSDn) at (13,17.75) {};
   
      \node[SD](V-ASD1) at (16,22.75) {};
   \draw[->,>=stealth,dotted,very thick] ([xshift=-80pt]V-ASD1)|-([yshift=25pt]BB);
      \node[SD](V-ASD2) at (16,20.75) {};     
      \draw[->,>=stealth,dotted, very thick] (V-ASD2)|-(BB);      
           \node[SD](V-ASDn) at (16,17.75) {};
         \draw[->,>=stealth,dotted, ,very thick] ([xshift=40pt]V-ASDn)|-([yshift=-25pt]BB);
         
         
     \node[align = center, text width = 22pt, thick]  (Vdots) at (14.5,19.5) {\Large{$\vdots$}}  ; 
     %
         %
    \node[human]  (Vn) at (14.5,17.75) {\Large{$V_n$}}  ;      
   \node at (VSDn) {{\textsf{VSD}}};
   \node at (V-ASDn) {{\textsf{ASD}}};
   \draw[->,>=stealth,line width=2pt] (7.2,25)--(7.2,25.8) -|([xshift=-60pt]Vn) ;
 \draw[-,>=stealth,dashed,line width=1pt,color=white] (7.2,25.1)--(7.2,25.8) -|([xshift=-60pt,yshift=-25pt]Vn) ;

         %         
    \node[human]  (V2) at (14.5,20.75) {\Large{$V_2$}}  ; 
   \node at (VSD2) {{\textsf{VSD}}};
   \node at (V-ASD2) {{\textsf{ASD}}};
  \draw[->,>=stealth,line width=2pt] (7,25.1)--(7,26) -|(V2) ;
    \draw[-,>=stealth,dashed,line width=1pt,color=white] (7,25.1)--(7,26) -|([yshift=-25pt]V2) ;
%         
   \node[human]  (V1) at (14.5,22.75) {{\Large{$V_1$}}}  ; 
   \node at (VSD1) {{\textsf{VSD}}};  
   \node at (V-ASD1) {{\textsf{ASD}}};
   \draw[->,>=stealth,line width=2pt] (6.8,25.1)--(6.8,26.3) -|([xshift=25pt]V1) ;
  \draw[-,>=stealth,dashed,line width=1pt,color=white] (6.8,25.1)--(6.8,26.3) -|([xshift=25pt,yshift=-25pt]V1) ;

 %

   %
\draw [->,>=stealth,line width=1pt] (V-ASD1)--(V1) -- (VSD1)-|([xshift=55pt]VC) ;
\draw [->,>=stealth,line width=1pt] (V-ASD2)--(V2) -- (VSD2)|-(VC) ;
\draw [->,>=stealth,line width=1pt] (V-ASDn)--(Vn)-- (VSDn)-|([xshift=60pt]VC) ;

 \node[draw,circle,text width=6pt,line width=0.8pt,fill = lightgray] (C1b) at (7.5,17.8) {};
 \node at (C1b) {1b} ;
 %
  \node[draw,circle,text width=6pt,line width=0.8pt,fill = lightgray] (C1a) at (7,25.7) {};
 \node at (C1a) {1a} ;
 %
  \node[draw,circle,text width=6pt,line width=0.8pt,fill = lightgray] (C3) at (5.5,17.8) {};
 \node at (C3) {4} ;
  %
  \node[draw,circle,text width=6pt,line width=0.8pt,fill = lightgray] (C2b) at (10.5,17.8) {};
 \node at (C2b) {3} ;
 %
  \node[draw,circle,text width=6pt,line width=0.8pt,fill = lightgray] (C41) at (13,16.3) {};
 \node at (C41) {5} ;
 %
  \node[draw,circle,text width=6pt,line width=0.8pt,fill = lightgray] (C42) at (14,16) {};
 \node at (C42) {5} ;
 %
  \node[draw,circle,text width=6pt,line width=0.8pt,fill = lightgray] (C4n) at (15,15.7) {};
 \node at (C4n) {5} ;
 
   \node[draw,circle,text width=6pt,line width=0.8pt,fill = lightgray] (C21) at (11.2,21.5) {};
 \node at (C21) {2} ;
 %
  \node[draw,circle,text width=6pt,line width=0.8pt,fill = lightgray] (C22) at (11.9,19.58) {};
 \node at (C22) {2} ;
 %
  \node[draw,circle,text width=6pt,line width=0.8pt,fill = lightgray] (C2n) at (11.5,18.45) {};
 \node at (C2n) {2} ;
\end{tikzpicture}
    \caption{The interaction among the entities in an e-voting execution. The dotted lines denote read-only access to the BB. The dashed arrows denote channels for voters' private inputs distribution. Annotation: (1a): distribution of voter's private inputs; (1b): posting pre-election BB data; (2): vote casting; (3): writing votes to BB; (4) posting post-election BB data and the election results; (5): auditing.}
        \label{interaction}
\end{figure}
An e-voting system $\Pi$ is a quintuple of algorithms and protocols  \textbf{(Setup, Cast, Tally, Result, Verify)}, that takes  voters' preferences as an input and aims to return a tally, the protocols specified as follows:
\begin{enumerate}
\item The interactive protocol \textbf{Setup} is executed by the $\ea$ and $T$. During the setup phase, $\ea$ generates $\Pi$'s public parameters $Pub$ (which include $P, V, U$) and the voters' secrets $s_1, \dots , s_n$. The part of the interactive protocol during which $\ea$ distributes secrets among voters is defined as \textbf{Registration}. At the same time, $T$ generates pre-election $\bb$ data and posts it on $\bb$.
\item The interactive protocol \textbf{Cast} is executed between three parties, the voter $V_l$, the $\bb$ and the $\voc$. During this interaction, the voter uses $\vsd$, his secret $s_l$ and an option $U_l$ to generate the ballot $b_l$ and sends this ballot to $\voc$. Upon successful termination, $\voc$ posts ballot $b_l$ to $\bb$ and the voter $V_l$ receives a receipt $\alpha_l$.
\item The algorithm \textbf{Result} is executed by $T$ and outputs the result $\tau$ for the election or returns $\perp$ in case such result is undefined.
\item  The algorithm \textbf{Verify} on input $\alpha,\tau$ outputs a value in $\{0,1\}$, where  $\alpha$ is a voter receipt (that corresponds to the voter's output from the \textbf{Cast} protocol).
\end{enumerate}
In some e-voting systems \textbf{Registration} part is omitted and voters are expected to receive their credentials via a secure channel such as post mail, polling place etc.\\

\begin{definition}[Correctness of e-voting system]
It is said that a system $\Pi$ has (perfect) correctness, if for any honest execution of any subset of not abstained voters that results in a public transcript $\tau$, where the voters $V_1, . . . , V_n$ cast votes for options $U_1, . . . , U_n$, it holds that $Result(\tau) = f(U_1,...,U_n)$, where $f(U_1,...,U_n)$ is the m-vector whose i-th location is equal to the number of times a candidate $P_i \in \{P_1,\dots, P_m\}$ was chosen in the candidate selections $U_1, . . . , U_n$.
\end{definition}
\subsection{Election process}
Any electronic voting procedure can be split into three general stages: Pre-election, Election and Post-election. For some specific e-voting schemes, the first or the last stage can be omitted.
\begin{enumerate}
\item Pre-election: $\ea$ generates public pre-election data and posts it on the $\bb$. Also, $\ea$ creates and distributes envelopes with private voters' information among all eligible voters. Meanwhile, $T$ obtains secret data that would be used for producing the result and posts its own pre-election data to the $\bb$.
\item Election: this stage typically is split into two parts:
\begin{enumerate}
\item Registration: A voter and $\ea$ engaged in an interaction during which the voter proofs to the $\ea$ his identity and obtains credentials for vote casting. In some e-voting systems, the registration part is omitted, and voters receive their credentials in envelopes via some secure channel. 
\item Vote casting: A voter starts an interaction with $\vsd$ using his credentials as proof of his eligibility to cast a vote for preferred candidates. Upon successful termination, $\voc$ receives a ballot from $\vsd$ and posts it on the $\bb$. 
\end{enumerate}
\item Post-election: After the election is closed, the election result is computed and announced by $T$.  If verification is supported, anyone may check the validity of the election procedure. 
\end{enumerate}

\subsection{E2E Verifiability}
E2E verifiability is a very strong level of security that allows voters  to detect that a malicious e-voting system tries to misrepresent the election outcome. \\

Three aspects of verifiability are usually distinguished:
\begin{enumerate}
\item \textit{Individual verifiability}: a voter can check that his ballot is counted correctly \cite{Chaum1981}.
\item \textit{Universal verifiability}: anyone can check that the election outcome is obtained from the ballots published on the $\bb$ \cite{Sako1995}.
\item \textit{End-to-end verifiability}: a voter can check that his vote was cast-as-intended and recorded-as-cast, also anyone can check that the ballots were tallied-as-recorded \cite{Neff2004},\cite{Chaum2004}.
\end{enumerate}

 In this work, we use the E2E-verifiability definition by Kiayias et al. \cite{Kiayias2015a},\cite{Kiayias2015}, that is given in Figure \ref{sec:e2e}. 
According to the definition, an adversary $\mathcal{A}$ can control all $\vsd$'s, $\ea$, $\voc$ and some fraction of voters, however, it still can not manipulate the results without a high detection probability.  The entities involved are $\bb$, $\vsd$ and $\ea$ that can be split on $\ea$, that involved on pre-election stage only, and $T$ that computes the results. The algorithm \textbf{Cast} is run interactively between tree parties the $\bb$, the voter $V_l$ and $T$ that uses $\vsd$ with the following inputs: public parameters $Pub$, voter's secret $s_l$ and voter's choice $\mathcal{U}_l$. Upon successful termination, $V_l$ obtains a receipt $\alpha_l$. The algorithm \textbf{Verify}($\tau,\alpha_l$) outputs a bit, based on voter's receipt $\alpha_l$ and the public transcript $\tau$. The algorithm \textbf{Result}($\tau$) given the public transcript $\tau$  outputs the result for the election or $\bot$ if the result is undefined.\\

We prefer the definition by Kiayias et al over the one given by K\"{u}sters et al \cite{Kusters2010}, because it is given in game-based settings. Moreover, this definition of E2E Verifiability does not require any additional assumptions, except for the existence of the $\bb$, therefore it is achievable in the standard model. Furthermore, there is an ideal functionality that captures the essential aspects of the E2E Verifiability \cite{idfunc}.  

\begin{figure}[h!]
 \fbox{\parbox{\textwidth}{
\underline{ E2E Verifiability Game $G^{\mathcal{A},\mathcal{E},d,\theta}_{E2E-Ver}(1^{\lambda},m,n)$}:
\begin{enumerate}
\item  $\mathcal{A}$ chooses a list of candidates $\mathcal{P} = \{P_1,\dots,P_m\}$, a set of voters $\mathcal{V} = \{V_1,\dots,V_n\}$ and the set of allowed candidate selections $\mathcal{U}$. It provides $\mathcal{C}$ with the sets $\mathcal{P}$,$\mathcal{V}$,$\mathcal{U}$ along with information $Pub$ and voter credentials $\{s_{l \in [n]}\}$. Throughout the game, $\mathcal{C}$ plays the role of the $\bb$.
\item  The adversary $\mathcal{A}$ and the challenger $\mathcal{C}$ engages in an interaction where  $\mathcal{A}$ schedules the \textbf{Cast} protocols of all voters. For each voter $V_l$  $\mathcal{A}$ can either completely control the voter or allow $\mathcal{C}$ to operate on their behalf, in which case $\mathcal{A}$ provides a candidate selection $\mathcal{U}_l$ to $\mathcal{C}$. Then, $\mathcal{C}$ engages with the adversary $\mathcal{A}$ in the \textbf{Cast} protocol so that  $\mathcal{A}$ plays the role of $\ea$. Provided the protocol terminates successfully, $\mathcal{C}$ obtains the receipt $\alpha_l$ on behalf of $V_l$.\\
 Let $\tilde{\mathcal{V}}$ be the set of honest voters (i.e., those controlled by $\mathcal{C}$) that terminated successfully. 
\item Finally, $\mathcal{A}$ posts the election transcript $\tau$ to the $\bb$.
\end{enumerate}
The game returns a bit which is 1 if and only if the following conditions hold true: \\
(i)  $|\tilde{\mathcal{V}}| \geq \theta$, (i.e., at least $\theta$ honest voters terminated).\\
(ii). $\forall l \in  [n]$: if $V_l \in  \tilde{\mathcal{V}}$, then \textbf{Verify}($\tau,\alpha_l$)=1 (i.e., the voters in $\tilde{\mathcal{V}}$ verify their ballot successfully).\\
and either one of the following two conditions: \\
~~~~(iii-a). If $\bot \ne \rangle \mathcal{U}_l \rangle _{V_l \in \mathcal{V}\setminus \tilde{\mathcal{V}}} \leftarrow \mathcal{E}(\tau, \{\alpha_l\} _{V_l \in   \tilde{\mathcal{V}}})$,\\
~~~~~~~then $d_1$(\textbf{Result}($\tau$),$f(\langle U_1,\dots,U_n \rangle)) \geq d$. \\
~~~~(iii-b). $\bot \leftarrow \mathcal{E}(\tau, \{\alpha_l\} _{V_l \in   \tilde{\mathcal{V}}})$.
 }}
\caption{E2E-verifiability by Kiayias et al.}
 \label{sec:e2e}
 \end{figure}
 



