%% This is an example first chapter.  You should put chapter/appendix that you
%% write into a separate file, and add a line \include{yourfilename} to
%% main.tex, where `yourfilename.tex' is the name of the chapter/appendix file.
%% You can process specific files by typing their names in at the 
%% \files=
%% prompt when you run the file main.tex through LaTeX.
\chapter{Conclusion}
We present the generalised model of e-voting system and analyse voter privacy with respect to different cases of collision among entities of this model. The specification of trusted entities splits the Voter Privacy into different scenarios. All meaningful cases of collusion fall into two scenarios: (1) $\ea$ and $\T$ are honest and (2) $\vsd$ and $\T$ are honest. It is possible to construct e-voting scheme, that is private with respect to only trusted $\vsd$, however, most known schemes require trusted $\T$ as well.  \\
 
The approach that we used for voter privacy in this work is the following: an honest voter is allowed to have only one perfectly hidden from adversarial eyes interaction and at the end he provides the adversary with (1) the real view of the result of this interaction and (2) a simulated one via an efficient algorithm $Sim$ called the simulator. The adversary $\mathcal{A}$ is allowed to observe a network trace of all interactions and play on behalf of corrupted entities and voters. This perfectly private interaction can be either an act of actually entering voter's preference into $\vsd$ or while a voter receives his credentials. If $\mathcal{A}$ has no advantage in distinguishing real and simulated view over a coin flip, the system is considered private.\\

Voter privacy suggests that voters are capable of casting their votes secretly and freely without letting adversarial parties to learn any information about their preferences. On the other hand, integrity is traditionally captured by the end-to-end (E2E) verifiability \cite{Benaloh2011} notion states that the voter can obtain a receipt at the end of the ballot casting procedure that is used for verifying that his vote was (1) cast as intended, (2) recorded as cast, and (3) tallied as recorded \cite{Kiayias2015a}. Furthermore, anyone should be able to verify that the election procedure is executed properly. It has been observed that voter privacy and E2E verifiability requirements inherently contradict each other at some point.  Therefore, there should exist the maximum level of privacy that is possible to achieve in any E2E verifiable e-voting system.\\

In this work, we perform a thorough and formal study on "locating" the critical contradiction point in the voter privacy-E2E verifiability tradeoff.  As part of our analysis, in chapter  \ref{strict} we introduce a strong privacy definition where voters are corrupted but an adversary is still unable to break privacy, denoted as strict privacy. We formally define strict voter privacy via a Voter Privacy game that is played between an adversary $\mathcal{A}$ and a challenger $\mathcal{C}$. According to the game rules, an adversary is allowed to define the election parameters, corrupt a number of entities, and act on behalf of all voters. As for E2E verifiability, we apply the definition given by Kiayias et al., according to which even when all election administrators are corrupted, they can not manipulate the results without a high detection probability.\\

Under this framework, we prove that strict privacy even in its weakest level contradicts end-to-end verifiability. However, any meaningful relaxation of the strict privacy definition leads to a notion of privacy that is feasible by some E2E verifiable e-voting system. \\

Also, we design a new e-voting system based on blind signature scheme, that captures the idea of anonymous voting, where everyone votes on behalf of an eligible group of voters. We argue that any system that keeps anonymous ballots is not E2E verifiable in the standard model. 