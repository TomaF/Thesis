% $Log: abstract.tex,v $
% Revision 1.1  93/05/14  14:56:25  starflt
% Initial revision
% 
% Revision 1.1  90/05/04  10:41:01  lwvanels
% Initial revision
% 
%
%% The text of your abstract and nothing else (other than comments) goes here.
%% It will be single-spaced and the rest of the text that is supposed to go on
%% the abstract page will be generated by the abstractpage environment.  This
%% file should be \input (not \include 'd) from cover.tex.
In an electronic voting (e-voting) execution, the voters engage in an interaction with the system by providing sensitive data such as their vote preference, authentication passwords, or personal data used for election auditing. All collected data should be processed in a way that election integrity and voter privacy are preserved at the best possible level. \\
Voter privacy suggests that voters are capable of casting their votes secretly and freely without letting adversarial parties to learn any information about their preferences. On the other hand, integrity is traditionally captured by  the end-to-end (E2E) verifiability notion states that the voter can obtain a receipt at the end of the ballot casting procedure that is used for verifying that his vote was (1) cast as intended, (2) recorded as cast, and (3) tallied as recorded. Furthermore, anyone should be able to verify that the election procedure is executed properly. It has been observed that voter privacy and E2E verifiability requirements inherently contradict each other at some point. Therefore, there should exist limits of privacy that is possible to achieve in any E2E verifiable e-voting system.\\
In this work, we perform a thorough and formal study on 'locating' the critical contradiction point in the voter privacy-E2E verifiability tradeoff. As part of this analysis, we introduce a strong privacy definition where voters are corrupted but an adversary is still unable to break privacy, denoted as  strict privacy. We formally define strict voter privacy via a Voter Privacy game that is played between an adversary  A and a challenger  C . According to the game rules, an adversary is allowed to define the election parameters, corrupt a number of entities, and act on behalf of all voters. As for we apply the E2E verifiability definition given by Kiayias et al. \cite{Kiayias2015}, according to which even when all election administrators are corrupted, they can not manipulate the results without a high detection probability.\\
Under this framework, we prove that strict privacy is the weakest level of privacy that contradicts end-to-end verifiability. Namely, any meaningful relaxation of the strict privacy definition, leads to a notion of privacy that is feasible by some E2E verifiable e-voting system.\\
Also, we have developed and implemented an e-voting scheme that is based on blind signature scheme approach for illustrating privacy limitations.
