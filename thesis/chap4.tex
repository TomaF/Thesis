%% This is an example first chapter.  You should put chapter/appendix that you
%% write into a separate file, and add a line \include{yourfilename} to
%% main.tex, where `yourfilename.tex' is the name of the chapter/appendix file.
%% You can process specific files by typing their names in at the 
%% \files=
%% prompt when you run the file main.tex through LaTeX.
\chapter{Blind signature e-voting scheme}
The idea of obtaining a proof of membership and voting anonymously is not new and had been considered the best possible solution for providing privacy to voters. However most approaches that based on group signatures \cite{Teranishi2009},\cite{Au2013} are traceable because it requires an additional tag that revivals the user's identity in case of exceeding the allowed number of cast votes.\\

A blind signature, on the other hand, makes it absolutely impossible to link a particular voter and his ballot based on published after or during an election information or leakage of authentication data.  It is on of the most popular techniques in e-voting schemes for ensuring confidentiality of the voter's ballot. Blind signatures have been used in many e-voting schemes designed last decades \cite{Ion2011},\cite{Ibrahim2003},\cite{Fujioka1993}. The main feature of the blind signature concept is that it allows to combine two contradictory ideas 1) the authorization of voters and 2)the anonymity of them. It allows to achieve such level of privacy that no one including $\ea$ and $\T$ can link the voter and his ballot. \\

We present an e-voting scheme that is based on the blind signature scheme and the ElGamal encryption, that is Voter Private with respect to trusted $\ea$ and $\T$ but not E2E Verifiable in the standard model. We argue that any e-voting scheme that keeps anonymous ballots is not E2E verifiable in the standard model.
\section{Presentation of the blind signature e-voting scheme}
Our system is a two-server web-based system. The first server - \textit{Signing server}  is used for \textbf{Registration} process and the second one  - \textit{$\voc$} for vote-casting.  In an election, the \textit{Signing server} mainly plays the roles of the $\ea$ but may aid the other parties.  Trustee $\T$ is a separate entity that has a secure channel to communicate with $\ea$, security of communication is supported by HTTPS. All the other parties are realised by Javascript running at the client side. The system uses additive ElGamal encryption, so the tally is done homomorphically. Currently, the blind signature  scheme supports approval type of voting s.t. $x$-out-of-$m$ type of option selection, where x is between 0 and $m$.
\section{Setup and parameters}
Throughout the paper, it is assumed that a group of $n$ voters $\mathcal{V} = \{V_1,\dots,V_n\}$ can choose as many candidates as they like among the set of $m$ candidates $\mathcal{P} = \{P_1,\dots,P_m\}$, $n$ and $m$ are both thought as polynomial functions of security parameter $\lambda$. Also, during the vote casting procedure no 'write-in's votes are allowed, the number of candidates $m$ is fixed. A voter may cast any vote including a blank or an invalid votes. \\

We denote by $M$ a strict upper bound on the number of votes any candidate can receive. In case when each voter has only one vote, M is a strict upper bound on the number of voters $n$.\\ 

It is assumed that the message space is $\mathbb{Z}_n$ for a suitable $n > M^m$, the $l_R$-bit randomizer size is  $\mathbb{Z}$. For NIZK proofs the cryptographic hash-function that outputs an $l_e$-bit number $e$ is used. In case of SHA-256  $l_e$ = 256. A security parameter $l_s$ is such that for any value $a$, its sum with a random $|a| + l_s$-bit number $a + r_a$ and this random number $r_a$ are indistinguishable. Article \cite{Groth2005} suggests to use $l_s = 80$, since this value is large enough to ignore the off chance that $|a+r| > |a|+l_s$.

\section{Syntax}
An e-voting system $\Pi$ is a quintuple of algorithms and protocols $\langle$ \textbf{Setup,Register, Cast,Tally, Result,Verify} $\rangle$ specified as follows:
\begin{enumerate}
\item \textbf{Setup}: The algorithm \textbf{Setup($1^{\lambda}, \mathcal{P}, \mathcal{V}, \mathcal{U}$)}  is executed by the $\ea$ and $\T$. During the setup phase $\ea$ generates $\Pi$'s public parameters $PubEA$ (which include $P, V, U, pkey, snPkey, params$), where $params$ are public election data, $pkey$ is the public key for encrypting votes and $snPkey$ is the public key that is used during the registration process for obtaining a blind signature. $\ea$ posts  $PubEA$  the on $\bb$ . Also, $\ea$ the voters' secrets $s_1, \dots , s_n$, that includes a unique salt seed $seed_i$ for AES encryption, a voter's login $login_i$ and password $password_i$ that are used for signing in. $\ea$ distributes secrets among all voters. At the same time, $\T$  obtains secret keys $skey$ and $snSkey$ for decrypting ballots and signing respectively and generates pre-election $\bb$ data $PubT$ and posts it on $\bb$. The public data is defined as $Pub = \langle PubEA, PubT \rangle $. 
\item  \textbf{Register}: The algorithm \textbf{Register($login_i,password_i,id_i$)}  is executed by each voter. A voter $V_i$ randomly chooses a $l_b$-bit string $x_i$ as his alias and a blinding factor $z_i$. He uses $x_i$, $z_i$ and $snPkey$ to construct the value $p_i = Hash(x_i)Enc(z_i;snPkey)$. The voter sends value $p_i$ to the $\ea$ along with his $login_i$, $password_i$ and proofs of identity $id_i$.
\item  \textbf{Cast}:  The interactive protocol \textbf{Cast($vote_i,x_i$)} is executed between tree parties, the voter $V_l$, the $\bb$and the $\voc$. During this interaction, the voter uses $\vsd$ to cast an encrypted ballot $b_i = \langle C, \alpha_i, \pi_i, hash_i, \rangle$, where $C=Enc(vote_i;r)$, $hash_i$ - HASH of entire ballot,  $\alpha_i$ and $\pi_i$  are non-interactive zero-knowledge proofs of eligibility and vote correctness respectively. Upon successful termination, $\voc$ posts ballot $b_i$ to $\bb$. The voter $V_l$ receives nothing.
\item  \textbf{Tally}: The interactive protocol \textbf{Tally($Pub$)}  can be executed by anyone due to a homomorphic property of the encryption scheme. The output is an encrypted sum of all valid ballots $\tau$ or $\perp$ in case all entries are invalid.
\item   \textbf{Result}: The algorithm  \textbf{Result($\tau, sKey$)} performs the decryption and outputs the result $R_{\tau}$ of the election and proofs of correct decryption $\pi_{\tau}$ or returns  $\perp$in case such result is undefined.
\item  \textbf{Verify}: The algorithm \textbf{Verify($b_i$)} outputs 1 if ballot $b_i$ is valid and 0 otherwise.
\end{enumerate}

\section{Building blocks}
\subsection{Blind signature}
In cryptography digital signature allows one to 'sign' massage in such a way that everyone can verify the validity of authentic signatures, but no one can forge (to produce a \textit{new} signature) signatures of new messages. A variation on basic digital signatures, known as blind digital signatures, include the additional requirement that a signer can 'sign' a message without knowing what the document contains. The concept of blind signing was introduced by David Chaum in \cite{Chaum1982}. \\

The global idea is to form a specially constructed message that hides a secret, obtain a signature for this message and then construct a valid signature for the secret. Suppose Alice wants Bob to sign a value $x$ with his private key $skey = (d,N)$ without learning $x$. To do so, Alice picks random blinding factor $z$ such that  $gcd(z,N) = 1$ and calculates $p=xz^e\bmod N $. She sends value $p$ to Bob. Bob signs $p^d\bmod N$ and sends it back to Alice. Alice calculates $p^d\bmod N z^{-1} = x^d\bmod N$ which is a signature for value $x$. Bob in not able to determine the secret $x$ on his own due to multiplication on unknown blinding factor. Blind signature can not be broken even with a help of a quantum computer.\\\\
\fbox{\begin{minipage}{30em}
\textit{Public input}:   $pkey = (e,N)$\\
\textit{Private input}:   $skey = (d,N)$\\\\
\textbf{Argument to sign:}\\
$p=xz^e\bmod N $\\
\textbf{Constructing signature}:\\
$p^d\bmod N z^{-1} =x^dz z^{-1} \bmod N =  x^d\bmod N$
 \end{minipage}}\\\\
 \textit{Security}:\\
 Security of the blind signature scheme relies on an information assumption and does not rely on a computational one, so it achieves information-theoretic security considered to be cryptanalytically unbreakable. That means that blind signature cannot be broken even if an adversary has unlimited computing power because the adversary simply does not have enough information to break the encryption.\\

Also, Chaum's blind signature satisfies the blindness property \cite{Chaum1982} and the non-forgeability \cite{Pointcheval1996},\cite{Abstract1997} of additional signatures, the former means that a signer can not  link the blinded message he signs and the original one except with negligible probability, and letter means that after getting $l$ signatures, it is infeasible to compute the $l+1$ signature.\\

Informally, unlinkability or blindness can be proven as follows. For any message $m$  there is a unique set of values $r_1,r_2,\dots,r_n$ that produces a set of blinded messages $m'_1,m'_2,\dots,m'_n$, with $mi_i  \equiv mr_i^e$. No set of values $r_1,r_2,\dots,r_n$ is any more probable than any other, hence the signer gets no information whether $m$ corresponds to $m'_1$ or $m'_2$ or whether it was one of the values signed at all.\\

Blind signature schemes can be implemented using a common public key signing schemes. One of the simplest blind signature schemes is based on RSA signing. \subsection{Homomorphic Integer Commitment and Homomorphic Cryptosystem}
There are only a few homomorphic integer commitment schemes \cite{Eiichiro1997},\cite{Damg},\cite{Groth2005a} and they are quite similar in structure. For the blind signature scheme  the following variant is offered.  A modulus $n$ as a product of two safe primes and random generators $g_1, \dots, g_k,h$ of $QR_n$ are chosen. In order to commit to integers $m_1,m_2, \dots, m_k$ using randomness $r$, it is necessary to compute $c = com( m_1,m_2, \dots, m_k; r) = g_1^{m_1}g_2^{m_2}\dots g_k^{m_k}h^r$. The commitment is statistically hiding if the randomness choice is $r \leftarrow \{0,1\}^{l_r}$.\\

\textit{Root extracting property}: When proving soundness and knowledge in protocols a root extraction property is needed. This property basically says that if a ciphertext raised to a nontrivial exponent encrypts 1, then the ciphertext itself encrypts 1 \cite{Groth2010}. For the homomorphic encryption generalisation of this property can be formalized as follows \cite{Groth2005}: \\

\fbox{\begin{minipage}{30em}
\textbf{Root extracting property:}\\
If there is a ciphertext $C$ and $e \neq  0$ so $|e| < l_e$ and $C^e = E(M;R)$, then it must be possible to find $\mu, \rho$ so $M = e\mu$,$R = e\rho$ and $C = E(\mu; \rho)$. 
\end{minipage}}\\

ElGamal encryption scheme  is an asymmetric key encryption algorithm for public-key cryptography with an implementation of Diffie-Hellman key distribution system  that was introduced  by Taher Elgamal in 1985 \cite{Elgamal1985}. The security of the scheme relies on the difficulty of computing the discrete logarithm over finite fields.  ElGamal encryption is semantically secure, has the root extraction property and admit threshold decryption  \cite{Groth2010}. \\

The use of homomorphic tallying scheme for e-voting is not new, it was originally introduced by Cramer et al. \cite{Cramer1997}  in 1997. From the E2E Verifiability point of view, the major benefit of using homomorphic tallying is that the proof of correct tallying and decryption is simplified and can be publicly verified if all encrypted votes are public \cite{Parsovs2016}. This means that tallied as recorder verification can be simplified and re-checked by anyone.  Moreover, the aggregated ciphertext and the corresponding plaintext cannot be linked back to individual votes. Publishing encrypted ballots can leas to long-term privacy risks, for example, the encryption may be broken after a few decades. One may mitigate the risks by limiting the access to $\bb$, where all encrypted ballots are stored. However, for the \textit{Blind signature scheme} we prefer to publish all encrypted ballots to achieve  public verifiability.\\

If tallied as recorder can be checked simply by anyone, then the only  verifiability feature still missing (except for cast as intended check) is to verify that all ballots are correctly formed and contains no more than one vote per candidate. This means that ballots should be submitted along with the zero-knowledge cryptographic proof of correctness. 
\subsection{Proving signature knowledge}
Non interactive zero knowledge (NIZK) proof  of signature knowledge a method by which one party (the prover) can successfully convince another party (the verifier) that the prover knows something without conveying any information apart from that fact.\\

Suppose there is a signature $\sigma$ for a hash value over message $x$ $HASH(x)$. Now a prover want to convince a verifier that he knows the signature without revealing value $\sigma$. The prover chooses a random variable $A$ and encrypt it using the public key $pkey = (e,N)$. Then he computes a product $S$ of the encrypted value and the signature raised in power of a challenge $c$. In the non-interactive proofs the challenge $c$ typically equal to the result of a hash function over the concatenation of all publicly known variables and arguments that is used for proving the statement. The NIZK argument is the challenge $c$ and the product $S$.\\\\
\fbox{\begin{minipage}{30em}
\textbf{Non-Interactive Zero-Knowledge Argument for proving signature knowledge:}\\\\
\textit{skey}:  (d,N)\\
\textit{pkey}:  (e,N)\\\\
Proof of the signature knowledge:\\
$NIZK[(x,\sigma): \sigma = x^d\bmod N]$\\\\
\textbf{Argument:}\\
$\sigma = HASH(x)^d\bmod N$\\
Choose random $A \in _R Z_N$\\
$R = A^e\bmod N$\\
$c = HASH(x||R)$\\
$S = A\sigma^c\bmod N$\\
The argument is: $(c,S)$\\\\
\textbf{Verify:}\\
$\hat{R} = \frac{S^e}{HASH(x)^c}\bmod N$\\
$\hat{c} = HASH(x||\hat{R})$\\
Verify $\hat{c} = c$
\end{minipage}}\\

Indeed if $\hat{c} = HASH(x|| \frac{S^e}{HASH(x)^c}\bmod N )= HASH(x|| A^e\bmod N ) =  HASH(x||R) = c$ then proofs of signature knowledge are valid.
\subsection{Proving that a ciphertext encrypts 0 or 1}

Each voter $V_i$ can vote for as many candidates from the set of all candidates as he likes. The voter specifies his choice by setting $a_j = 1$ if he wishes to vote for candidate $j$ and $a_j = 0$ if he does not. The plaintext vote is $VOTE_i =\sum_{j=0}^{m-1}a_jM^j$. The voter encrypts this to get a ciphertext $C = E(\sum_{j=0}^{m-1}a_jM^j;R)$. He now needs to prove that indeed the plaintext is on the right form so $a_j = 0 \vee 1$ for all $j \in [0,m-1]$. \\

A prover commit to values $a_0, \dots , a_{m-1}$. In order to prove that all hidden $a_j \in \{0, 1\}$  the fact that $x^2 \geq x$ for any integer and equality holds only for $x = 0$ or $x = 1$ is used. This means that if the prover want to convince the verifier that  all $a_j$'s belong to $\{0, 1\}$ he needs to convince him that $\Delta = \sum_{j=0}^{m-1}(a_j^2  - a_j) = 0$.\\

The variables $a_j$ and $\Delta$ are hidden using the standard techniques $\boxed{a_j} = ea_j +r_{a_j}$  and $\boxed{\Delta} = e\Delta+r_\Delta$. In the verification, the verifier construct the same value $\boxed{\Delta} = \sum_{j=0}^{m-1}(\boxed{a_j}^2  - e\boxed{a_j})$. The left side of this equation  is a degree 1 polynomial in $e$ while the right side is a degree 2 polynomial in $e$. With overwhelming probability over $e$ this implies that the value the prover committed to $\Delta =0$, which is exactly what needs to be proven.\\
 
 The rest of the NIZK argument are a proof of knowledge of the plaintext $V$ and a proof that this plaintext has been properly constructed with values $a_j$'s that the prover committed to.\\\\
\fbox{\begin{minipage}{30em}
\textbf{Non-Interactive Zero-Knowledge Argument for Correctness of the encrypted vote:}\\\\
The proof of correctness:\\
$NIZK[(v,\rho,a_0,\dots, a_{m-1}): C = Encr(v;\rho) \text{ and   } v = \sum_{j=0}^{m-1}a_jM^j $\\
$\text{ and } \sum_{j=0}^{m-1}(a_j^2 - a_j) = 0]$\\\\
\textbf{Argument:}\\
Choose $r_{a_0},\dots r_{a_{m-1}} \leftarrow \{0,1\}^{1+l_s+l_e}$ and let $\Delta = \sum_{j=0}^{m-1}(2a_j -1)r_{a_j}$ \\
Choose $r \leftarrow \{0,1\}^{l_r}$ and set $c = com(a_0,\dots,a_{m-1},\delta;r)$\\
Choose $r_r \leftarrow \{0,1\}^{l_r+l_s + l_e}$ and set $c_r  = com(r_{a_0},\dots,r_{a_{m-1}},\sum_{j=0}^{m-1}r_{a_j}^2;r_r)$\\
$R_v = \sum{j=0}^{m-1}r_{a_j}M^j$, choose $R_R \leftarrow \{0,1\}^{l_R +l_e+l_s}$ and set $C_R = Encr(R_V;R_R)$\\
Compute a challenge $e \leftarrow hash(C,C_R,c,c_r)$\\
Set $\boxed{R} = eR + R_R$. Set $\boxed{a_j} = ea_j + r_{a_j}$ and $\boxed{r} = er + r_r$\\
The argument is: $C_R,c,c_r, \boxed{R},\boxed{a_0},\dots,\boxed{a_{m-1}},\boxed{r}$\\\\
 \textbf{Verification:}\\
 Compute $e$ as above.\\
 Define $\boxed{V} = \sum_{j=0}^{m-1}\boxed{a_j}M^j$ and \\
 $\boxed{\Delta} = \sum_{j=0}^{m-1}(\boxed{a_j}^2 - e\boxed{a_j}).$\\
 Verify $C^eC_R = Encr(\boxed{V};\boxed{R})$ and $c^ec_r = com(\boxed{a_0},\dots,\boxed{a_{m-1}},\boxed{\Delta};\boxed{r})$ 
\end{minipage}}\\

The complete prove of completeness, zero-knowledge as well as soundness and knowledge of the  described NIZK argument  can be found here \cite{Groth2005}.
\section{System Design}
Here is the description of a blind signature e-voting system for r-out-of-m elections, where $0\leq r\leq m$.\\
The Blind signature scheme:\\ 
\textbf{Setup ($1^{\lambda}, \mathcal{P}, \mathcal{V}, \mathcal{U}$).}\\\\
Let ($GenBL, EncrB, SignBL)$ be the PPT algorithms that constitutive the RSA blind signature scheme,  $PRG_{str}$  - a function for pseudo random string generation, $PRG_{prime}$  - a function for pseudo-random prime number generation and $(Gen, Encr, Decr)$ - PPT algorithms for the ELGamal encryption scheme. The EA runs $GenBL(Param, 1^{\lambda})$ to generate the blind signature scheme keys $(bsk, bpk)$ and $Gen(Param, 1^{\lambda})$ to generate ELGamal keys $(sk,pk)$. Public keys $bpk,pk$ and functions $EncrB,Encr, PRG_{str}, PRG_{prime}$ are posted on the BB.\\
Then, for every voter $V_l$, where $l \in [n]$, EA runs $PRG_{str}(1^{\lambda})$ function to generate random string $seed_l$.\\\\
\textbf{Registration}\\\\
Let $(GenAES, EncrAES, DecrAES)$ be the publicly known PPT algorithms that implements AES encryption scheme and $Hash$ - hash algorithm.  \\\\
Every voter $V_l$ completes the following procedure: \\
--  uses the published on the BB function $PRG_{str}(1^{\lambda})$ to generate his alias $x_l$ and  $PRG_{prime}(bpk)$ to generate a blinding factor $z_l$. \\
-- calculates value $p_l = Hash(x_l)EncrB(z_l)$.\\
--  chooses his secret password $password_l$ and runs  $ GenAES(password_l, s_l)$ to generate his key for symmetric encryption $key_l$, where $s_l = PRG_{str}(seed_l)$.\\
--  encrypts $x_l$ and $z_l$ by running  $EncrAES(key_l,x_l)$ and $EncrAES(key_l,z_l)$ respectively to get encrypted values $\hat{x_l},\hat{z_l}$\\
-- sends $p_l,\hat{x_l},\hat{z_l}$ to the EA\\\\
 EA:\\
Upon receiving $p_l,\hat{x_l},\hat{z_l}$ from a voter, EA posts all information to the BB.\\
When registration is closed, EA runs  $SignBL(bsk, p_l)$ for every entry in the BB and post the result in the corresponding line as $p^{sign}_l$.\\\\
\textbf{Cast:}\\
Let $e_l = (e_{1_l},e_{2_l},\dots,e_{m_l})$ be the characteristic vector corresponding to the voter's selection, where $e_{j_l}=1$ if the option $opt_j$ is selected by the voter $V_l$.\\ 
Every voter $V_l$ completes the following procedure:\\
-- gets all information from the BB - $\{p_l,\hat{x_l},\hat{z_l}\}$.\\
-- finds his entry and decrypts $x_l$ and $z_l$ by running $DecrAES(key_l,\hat{x_l})$ and $DecrAES(key_l,\hat{z_l})$ respectively, where $key_l =  GenAES(password_l, s_l)$ and  $s_l = PRG_{str}(seed_l)$\\
-- computes  $\sigma_l$ - signature for $x_l$  by calculating  $p^{signed}_lz_l^{-1}$\\
-- computes $\Pi_l$ -- NIZK proof of signature knowledge .\\
-- chooses his vote-option $e_{j_l}$ and writes the corresponding characteristic vector $e_l$\\
-- for $j \in [m]$ compute $c_{j_l} =  Encr(pk,e_{j_l})$\\
-- computes NIZK proofs $\pi_{j_l}$ that each $c_{j_l}$ is an encryption of 1 or 0.\\
-- sends $b_l = (x_l, \sigma_l, \Pi_l, c_l, \pi_l)$ to the EA.\\
-- keeps $x_l$ as receipt.\\
-- If EA accept $b_l$, protocol terminates successfully.\\ 
*Optional: every voter can export the randomness to check that the ballot was cast as intended.\\\\
Upon receiving $(x_l, \sigma_l, \Pi_l, c_l, \pi_l)$ from a voter, EA checks NIZK proofs and, if it's valid, accepts the ballot and posts all information to the BB.\\\\
\textbf{Tally: }\\
After election is closed, EA computes $\mathcal{C}$ -- the sum of all $c_l$ and runs $Decr(sk, \mathcal{C})$ to decrypt Tally $\tau$. EA posts the $\tau$ along with the proof of tally correctness $Proof$.\\\\
\textbf{Result:} \\
result is straightforward.\\\\
\textbf{Verify:} \\
The algorithm returns 1 only if the following checks are true:\\
--  exported randomness is correct or voter choose not to check.\\
-- there is ballot with $x_l$\\
-- all $\Pi_l, \pi_l$ are valid\\
-- number of ballots less or equal to the number of registered voters\\
-- $Proof$ is correct\\
-- sum of all scores at $\tau$ are less than or equal to ballot numbers\\\\
\section{E2E Verifiability}
In the E2E verifiability proof, only $\bb$ is assumed to be honest. The rest components of the election server are controlled by the adversary and its is allowed to change the content on the BB arbitrarily before the \textbf{Result} protocol starts. However, it can be assumed that validity of all NIZK proofs on the $\bb$ can be checked by anyone as well as the result of the \textbf{Tally} protocol execution, since $Tally$ is computed homomorphically and this computation doesn't require any secret information. To prove that the blind signature scheme is E2E verifiable, we first construct a vote extractor $\mathcal{E}$:\\
 
$\mathcal{E}$ has input $\tau$ and the set of receipts  $\{x_l\}_{V_l \in \tilde{\mathcal{V}}}$ where $\tilde{\mathcal{V}}$ is the set of the honest voters that voted successfully.  If $Result(\tau) = \perp$ (i.e., the transcript is not meaningful), then $\mathcal{E}$ outputs $\perp$. Otherwise, $\mathcal{E}$ (arbitrarily) arranges the voters in $\mathcal{V} \backslash \tilde{\mathcal{V}}$ and the tags not included in $\{x_l\}_{V_l \in \tilde{\mathcal{V}}}$ as $\langle V_l^{\mathcal{E}} \rangle_{l \in  [n - |\tilde{\mathcal{V}}|]}$ and $\langle tag_l^{\mathcal{E}} \rangle_{l \in  [n - |\tilde{\mathcal{V}}|]}$ respectively. Next, for every $l \in [n - |\tilde{\mathcal{V}}|]$:\\
\begin{enumerate}
\item  $\mathcal{E}$ finds at the BB entry with   $x_l = tag_l^{\mathcal{E}}$ and brute-force the corresponding ELGamal cipher to open the selected candidate $\mathcal{P}_l^{\mathcal{E}}$. If $\mathcal{P}_l^{\mathcal{E}}$ is the valid candidate's selection, then $\mathcal{E}$ sets $\mathcal{U}_l^{\mathcal{E}} = \{\mathcal{P}_l^{\mathcal{E}}\}$. Otherwise it inputs $\perp$.
\end{enumerate}
Finally $\mathcal{E}$ outputs  $\langle \mathcal{U}_l^{\mathcal{E}} \rangle_{V_l \in \mathcal{V} \backslash \tilde{\mathcal{V}}  }$
Finally $\mathcal{E}$ outputs  $\langle \mathcal{U}_l^{\mathcal{E}} \rangle_{V_l \in \mathcal{V} \backslash \tilde{\mathcal{V}}  }$\\

According to the definition of E2E verifiability, an adversary $\mathcal{A}$ wins the game $G^{\mathcal{A} ,\mathcal{E} ,d,\theta}_{\text{e2e-ver}} (1^{\lambda}, n,m,t)$ and breaks E2E verifiability if it allows at least $\theta$ honest voters to cast their votes successfully and achieves tally deviation $d$. \\

All NIZK proofs are perfectly sound and the \textbf{Tally} protocol is completely transparent and can be repeated and checked by anyone. Thus, $\mathcal{A}$ can not misinterpret results or encode more than one vote for a particular candidate. The only possible way for $\mathcal{A}$ to cheat is to somehow modify honest votes' intents. \\

\textbf{Modification attack}: $\mathcal{A}$, who controls the whole system, may attempt to modify a ballot by decrypting it, learning $x$, creating a new signature and voting for another candidate, however since each voter has the hash of the whole his ballot as receipt, malicious $\ea$ would be caught. \\

\textbf{Clash attacks}: This attack is statistically improbable, since $x$ is unknown for $\mathcal{A}$ large string picked at random by a voter. If  $\mathcal{A}$ modifies the pseudo-random generator in a way that a number of voters gets the same not random string, it is easily detectable since  this part is done on the client side and the code is available for audition. \\

However $\ea$ can use abstain voters, who registered but did not cast their votes,  to create additional ballots and cause deviation. It can use $y_{reg} - y_{voted} -1$  additional votes. Abstain voter can not prove anything since he doesn't know whether he is the only one who abstains or not. This attack can be detected only if every registered voter is assigned a specific entry in $\bb$. Unfortunately,  assigning every voter ballot id makes this system linkable.\\

There is a tradeoff: either $\ea$ should be able to link an individual voter with his ballot or $\ea$ can vote on behalf of abstain voters and no-one would detect it as long as number of additional ballots strictly less than the number of abstain voters. This attack applies to any e-voting systems that allows voters to vote anonymously.  

\section{Implementation}
The blind signature scheme system is an open source web-based public auditable e-voting system. The system consists of two main servers: \textit{Signing Server} (SS) and \textit{Ballot Server} (BS). The server side is written on Node.js, that uses event-based server execution procedure rather than the multithreaded execution in PHP. Basically, Node.js utilise the same even-based  asynchronous technology as JavaScript. Node.js has proven itself capable of handling millions of concurrent connections and it is cross-platform. The source code is available on Github \cite{git}.\\

For the testing purpose we used self-singed SSL certificates for BS and SS. Certificate are used to prevent man-in-the-middle attacks and ensure that the certificate holder is really who he claims to be and usually signed by a trusted certificate authority (CA). However in the BSS proof of concept instead of obtaining real certificate signed by CA we used the openssl toolkit for generated self-sign certificates. This toolkit allows to generate a 1024-bit private key using RSA encryption and issue a certificate signing request.\\ 

Both servers SS and BS are HTTPS servers with self-signed certificate. The former server lets users to sign up for an election, check list of registered voters and sign a random credentials blindly. The later server allows users to cast their votes, check any vote in database, find their own vote based on QR code, decrypt all votes and show all cast votes.\\

All cryptographic primitives are implemented using Forge Javascript Cryptography Library. The Forge library implements TLS protocol and the set of cryptographic algorithms: AES (CBC), RSA, MD5, SHA-1, SHA-256 message digests, HMAC support, PKCS\#5 password-based key-derivation. \\

$\bb$ is implemented as an MySQL database that contains two tables: \textit{list} and \textit{log}. The description of tables is given below:\\

\begin{table}[h!]
\centering
\caption {list}
\begin{tabular}{|l|l|l|l|l|l|}
\hline
 Field & Type&     Null & Key & Default & Extra \\ 
 \hline 
 number &  int(5) & NO   & PRI  & NULL &  auto\_increment \\
id& varchar(300) & NO   &  &  NULL &\\
x& varchar(300) & YES  &  & NULL  & \\
z& varchar(300) & YES  &  & NULL  & \\
p& varchar(300) & NO  &  &  NULL & \\
psign&  varchar(300) & YES  &  & NULL  & \\
submission\_time& timestamp & NO  &  &  NULL & CURRENT\_TIMESTAMP\\
\hline
\end{tabular}
\end{table}
\begin{table}[!htb]
\centering
\caption {log}
\begin{tabular}{|l|l|l|l|l|l|}
\hline
 Field & Type&     Null & Key & Default & Extra \\ 
 \hline 
id &  int(11) & NO   & PRI  & NULL &  auto\_increment \\
x& varchar(300) & NO  &  & NULL  & \\
s& varchar(300) & NO  &  & NULL  & \\
c& varchar(300) & NO  &  & NULL  & \\
submission\_time& timestamp & NO  &  &  NULL & CURRENT\_TIMESTAMP\\
vote&  varchar(300) & YES  &  & NULL  & \\
decrvote&  varchar(300) & YES  &  & NULL  & \\
hash&  varchar(300) & NO  &  & NULL  & \\
\hline
\end{tabular}
\end{table}
